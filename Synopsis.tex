\documentclass[12pt]{article}
\usepackage[margin=1.0 in]{geometry}
\addtolength{\topmargin}{.25in}
\usepackage[utf8x]{inputenc}  
\usepackage{amsmath}
\usepackage{calc}
\usepackage{array}
\usepackage{amssymb}
\usepackage{tikz}
\usepackage{pgfgantt}
\usepackage{hyperref}
\usepackage{graphicx}
\usepackage{upquote}
\newcommand{\HRule}{\rule{\linewidth}{0.5mm}}
\usepackage{hyperref}
\newcommand{\Green}{\tikz\draw[green,fill=green] (0,0) circle (1 ex);}
\newcommand{\Lime }{\tikz\draw[brown,fill=brown] (0,0) circle (1 ex);}
\newcommand{\Blue}{\tikz\draw[blue,fill=blue] (0,0) circle (1 ex);}
\newcommand{\Yellow}{\tikz\draw[yellow,fill=yellow] (0,0) circle (1 ex);}
\newcommand{\Red}{\tikz\draw[red,fill=red] (0,0) circle (1 ex);}
\renewcommand*\contentsname{Indholdsfortegnelse}
\definecolor{listinggray}{gray}{0.9}
\usepackage{listings}
\lstset{
	language=,
	literate=
		{æ}{{\ae}}1
		{ø}{{\o}}1
		{å}{{\aa}}1
		{Æ}{{\AE}}1
		{Ø}{{\O}}1
		{Å}{{\AA}}1,
	backgroundcolor=\color{listinggray},
	tabsize=3,
	rulecolor=,
	basicstyle=\scriptsize,
	upquote=true,
	aboveskip={1.5\baselineskip},
	columns=fixed,
	showstringspaces=false,
	extendedchars=true,
	breaklines=true,
	prebreak =\raisebox{0ex}[0ex][0ex]{\ensuremath{\hookleftarrow}},
	frame=single,
	showtabs=false,
	showspaces=false,
	showstringspaces=false,
	identifierstyle=\ttfamily,
	keywordstyle=\color[rgb]{0,0,1},
	commentstyle=\color[rgb]{0.133,0.545,0.133},
	stringstyle=\color[rgb]{0.627,0.126,0.941},
}
\begin{document}
\begin{titlepage}
\begin{center}

\textsc{\Large Bachelor Thesis \\ Optimized pattern matching in genomic data\\[0.3cm]}
\HRule \\[0.4cm]
{ \LARGE \bfseries Synopsis}\\[0.4cm]
\HRule \\[1.2cm]
\textsc{\large Martin Westh Petersen - mqt967 \\ Kasper Myrtue - vkl275}\\[1.0cm]
\end{center}
\begin{center}
\vfill
{\large 23. Februar 2015}
\end{center}
\end{titlepage}
\section{Problem definition}
Is it possible create a program with the same core functionality as scan\_for\_mathces, but with an equal or
increased performance every time?
\section{Limitations}
We will be implementing the core functionality of scan\_for\_matches which we define as follows:
\begin{itemize}
\item Constructing literal pattern units (e.g. "AGUUG") that are used for finding a specific sub-sequence in the 
database sequence.
\item Constructing ambiguous pattern units (e.g. "4..8") that match any sub-sequence with a possibility for a 
flexible range.
\item Combining pattern units to a full pattern that defines the full search criteria when scanning the database sequence.
\item Allowing any pattern unit a specified number of insertions, deletion and mismatches (e.g. "AGGUAAA[2,0,3]").
\item Defining variables for referencing pattern units (e.g. "p1=5..6") that can be used 
to find related patterns that are unknown before search (e.g. p1=5..6 p1[1,0,0])
\item The $\sim$ symbol can be inserted in front of any pattern unit to indicate that we are looking for the reversed
complement of that pattern unit.
\end{itemize}
Beside these already existing features, we will be implementing a more optimized way of searching for matches, given
complex patterns consisting of multiple pattern units. E.g. the order of which the different pattern units are searched
can be optimized to increase performance instead of going through the pattern units from one end to the other. \\ \\
In case of excess time, secondary features may be implemented: 
\begin{itemize}
\item Logical "or" between patterns (e.g. "(AUUG $|$ AGGG)") that matches either of the sub-sequences.
\item A possibility for defining custom "pairing rules" (e.g. "r1={au,ua,gc,cg,ga,ag}") that can be used for 
for defining allowances when comparing a reversed complement pattern unit (e.g. "r1$\sim$p1").
\item An analysis of the complexity and running time of the program.
\end{itemize}
Every feature from scan\_for\_mathces beyond what has been mentioned will not be implemented.
\section{Motive}
Pattern matching functionality for strings in genomic data is very useful, 
but requires a good performance due to huge amounts of data. 
Scan\_for\_matches serves this purpose, but big improvements in performance can be made.
On top that, the code is poorly documented, lacks version control and the code is hard to read and maintain.
\section{Tasks and schedule}

\begin{figure}
\caption{gantchart of our schedule. The blue line indicates midway report, the red line indicates the final report.}
Below is a list of the tasks that this project consists of:
\begin{itemize}
\item Why is scan\_for\_matches fast despite the use of a backtracking algorithm? This research task is about 
reading and understanding the vital parts of the scan\_for\_matches code so that we understand the overall
ideas and algorithms, and can reuse them if needed.
\end{itemize}
\begin{figure}[h!]
\begin{ganttchart}[
hgrid=true,
vgrid={*9{white},*1{blue},*6{white},*1{red}}
]{1}{18}
\gantttitle{2015}{18} \\
\gantttitlelist{7,...,24}{1} \\
%\ganttmilestone{Midway report}{10} \ganttnewline
%\ganttmilestone{Final report}{17} \ganttnewline
\ganttbar{Research}{1}{4} \\
\ganttbar{Design}{2}{8} \\
\ganttbar{Implementation}{4}{15} \\
\ganttbar{Testing}{11}{15} \\
\ganttbar{Documentation}{4}{15} \\
\ganttbar{Report}{2}{17} \\
\end{ganttchart}
\caption{Schedule}
\end{figure}

\begin{itemize}
\item How to load data effectively into the pattern searcher
\item How to store data effectively when searching for a pattern
\item How to output data in the most useful way for further use
\item How to input patterns (commandline library for python or other)
\item What makes Scan\_for\_matches so effective
\item Is parallelism a valid approach for this type of search
\item design a string search for specific strings
\end{itemize}
\end{document}
