\documentclass[12pt]{article}
\usepackage[margin=1.0 in]{geometry}
\addtolength{\topmargin}{.25in}
\usepackage[utf8x]{inputenc}  
\usepackage{amsmath}
\usepackage{calc}
\usepackage{array}
\usepackage{amssymb}
\usepackage{tikz}
%\usepackage{pgfgantt}
\usepackage{hyperref}
\usepackage{graphicx}
\usepackage{upquote}
\newcommand{\HRule}{\rule{\linewidth}{0.5mm}}
\usepackage{hyperref}
\newcommand{\Green}{\tikz\draw[green,fill=green] (0,0) circle (1 ex);}
\newcommand{\Lime }{\tikz\draw[brown,fill=brown] (0,0) circle (1 ex);}
\newcommand{\Blue}{\tikz\draw[blue,fill=blue] (0,0) circle (1 ex);}
\newcommand{\Yellow}{\tikz\draw[yellow,fill=yellow] (0,0) circle (1 ex);}
\newcommand{\Red}{\tikz\draw[red,fill=red] (0,0) circle (1 ex);}
\renewcommand*\contentsname{Indholdsfortegnelse}
\definecolor{listinggray}{gray}{0.9}
\usepackage{listings}
\lstset{
	language=,
	literate=
		{æ}{{\ae}}1
		{ø}{{\o}}1
		{å}{{\aa}}1
		{Æ}{{\AE}}1
		{Ø}{{\O}}1
		{Å}{{\AA}}1,
	backgroundcolor=\color{listinggray},
	tabsize=3,
	rulecolor=,
	basicstyle=\scriptsize,
	upquote=true,
	aboveskip={1.5\baselineskip},
	columns=fixed,
	showstringspaces=false,
	extendedchars=true,
	breaklines=true,
	prebreak =\raisebox{0ex}[0ex][0ex]{\ensuremath{\hookleftarrow}},
	frame=single,
	showtabs=false,
	showspaces=false,
	showstringspaces=false,
	identifierstyle=\ttfamily,
	keywordstyle=\color[rgb]{0,0,1},
	commentstyle=\color[rgb]{0.133,0.545,0.133},
	stringstyle=\color[rgb]{0.627,0.126,0.941},
}
\begin{document}
\begin{titlepage}
\begin{center}

\textsc{\Large Bachelor Thesis \\ Optimized pattern matching in genomic data\\[0.3cm]}
\HRule \\[0.4cm]
{ \LARGE \bfseries Report}\\[0.4cm]
\HRule \\[1.2cm]
\textsc{\large Martin Westh Petersen - mqt967 \\ Kasper Myrtue - vkl275}\\[1.0cm]
\end{center}
\begin{center}
\vfill
{\large 20. April 2015}
\end{center}
\end{titlepage}
\tableofcontents
\section{Program structure}
A list of actions our program should do in order to execute a pattern search:
\begin{itemize}
\item Read and parse the input line. E.g. "scanFM 'ATTGCCCC[0,1,2]' 'data.txt'". Possibility of writing "$->$ 'output.txt'"
which results in the matches not being displayed in the terminal but written the the specified file.
\item Parse the pattern into units and save them as different types (objects of different classes that inherit
from a common PUNIT class), e.g. EXACT\_PUNIT, AMBI\_PUNIT etc.
\item Choose the order of which to search for the patterns and create a state that readies for this search, for example a 
list of the punits in correct order with some way of keeping track of the different positions the punits have to with
respect to each other.
\item Search for the punits in the order chosen by simply calling a ".search()" method for each PUNIT-object.
The PUNIT-object's search method invoked is unique for each different type of PUNIT, and returns either True or False.
If the search for each PUNIT returns True the match is saved.
\item The saved matches are either displayed in the terminal or written in a file, depending on the call of scanFM. 
\end{itemize}
Types of PUNITs that we need
\begin{itemize}
\item EXACT - A PUNIT of this type consists of either of the letters 'A', 'C', 'G' and 'T' or any number of the wildcards.
\item  
\end{itemize}
\end{document}
